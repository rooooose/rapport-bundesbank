\section{Sujet et place dans le service}
\subsection{Sujet}
\subsubsection{Sujet défini avant mon arrivée}

Avant le début de mon stage, il était prévu que je travaille sur un projet innovant visant à développer une preuve de concept pour extraire des informations numériques sur les émissions de carbone à partir de rapports de durabilité en pdf. 
Pour ce faire, nous devions utiliser des technique de Natural Language Processing en python.\\

~Le soutien aux tâches en cours afin de connaître toutes les fonctions clés de l'équipe faisait également partie intégrante du stage. 
Cela aurait pu inclure la documentation des données, la liaison et la comparaison des ensembles de données pour la gestion de la qualité des données et la participation aux groupes de travail internes et externes.


\subsubsection{Sujet réel}

Mes tâches en réalité étaient plus variées et sur divers projets :
\begin{itemize}
    \item Crawling de résultats de recherche Google pour récupérer les rapports de durabilité des entreprises, et extraction d'informations des PDF
    \item Optimisation de code Python pour la comparaison de gros jeux de données
    \item Rapprochement de jeux de données provenant de différentes sources avec R (analyses)
    \item Génération automatisée d'un rapport à partir de résultats stockés dans un fichier csv
    \item Projet de recherche NLP pour trouver quels datasets et méthodologies sont utilisé(e)s dans quels papiers de recherche
\end{itemize}


\pagebreak

\subsection{Fonction occupée dans le service}

Ma fonction dans le SFDH, et plus généralement dans la DSZ était celle de stagiaire Data Scientist, aux côtés d'un autre stagiaire occupant exactement la même fonction.
Comme détaillé dans le sujet, des tâches diverses en rapport avec les données m'ont été assignées sur de nombreux projets.
\\

Le point commun entre toutes mes tâches était qu'à partir d'un gros volume de données, je devais en créer de nouvelles qui soient utilisables pour le but souhaité et compréhensibles par les différents corps de métier présents à la Bundesbank.
Autrement dit, il s'agissait de tâches de valorisation des données. Pour cela, je passais par diverses méthodes de traitement de données à l'aide de Python ou R.
J'ai parfois effectué le travail de collecte de données, notamment pour le projet GAIA (crawling).
\\

~Les projets ne se restreignaient pas au SFDH, ils concernaient aussi (et pour la plupart) d'autres services de la DSZ, notamment ceux dans la recherche :
Certaines tâches m'ont alors été attribuées non pas par mon tuteur, mais par d'autres collègues d'autres services.