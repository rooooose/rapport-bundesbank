\section*{Conclusion} \markboth{Conclusion}{}
\phantomsection
\addcontentsline{toc}{section}{Conclusion}
\thispagestyle{noTitledHeader}

J’ai beaucoup appris sur le traitement des exceptions et l’écriture / lecture d’un fichier / utiliser l'API Dropbox, mais surtout faire un Google Crawler

Durant ce stage chez Gaea21, j'aurai donc réussi avec mon équipe à terminer les principales fonctionnalités de la version 0 du site Répertoire Vert.

Ainsi, une entreprise peut à présent :
\begin{itemize}
    \item S'inscrire et se connecter
    \item Visualiser son profil et modifier ses informations
    \item Désactiver son compte
    \item Ajouter des produits et/ou services
    \item Visualiser ses produits/services, les supprimer et les modifier
    \item Visualiser ses statistiques, comme le nombre de clics sur ses produits, sans bug
    \item Parcourir les différentes catégories et sous-catégories, et voir la liste des entreprises appartenant à chaque sous-catégorie.
    \item Visualiser les profils des autres entreprises ainsi que leurs produits/services
\end{itemize}

Mes compétences se sont nettement améliorées en Git, que je sais maintenant utiliser en ligne de commande.
Je sais désormais mettre un site en production, et coder avec Symfony et ReactJS.\\

J'ai pu apprendre à gérer une équipe et un projet avec des deadlines serrées, en utilisant des outils de gestion. 

Mes collègues étaient pour la plupart plus âgés que moi, mais j'étais la plus ancienne sur le projet (à cause d'un turn-over important). 
De ce fait, je connaissais assez le projet pour pouvoir les accompagner. Mais la différence d'âge ne se ressentait pas, travailler avec eux était très fluide.
Il y avait toujours un climat d'entraide et beaucoup de communication dans l'équipe.

Les réunions journalières avec mon tuteur et l'équipe me permettait de suivre leur avancée, et additionnellement si besoin, nous nous appelions avec un ou plusieurs membres, par exemple s'ils avaient besoin d'aide ou que quelque chose devait être rectifié.
Nous communiquions beaucoup par Skype.
Aussi, je mettais en commun notre travail plusieurs fois par semaine, ce qui me permettait de bien suivre la progression du projet.

L'outil de gestion que j'utilisais pour planifier était un tableau Google Sheet présentant les tâches, leur difficulté, et leur durée estimée que je remplissais avec mon équipe.\\

Avoir une certaine responsabilité était très formateur et m'a donné une idée du métier de Lead Programmer, qui me plairait d'ailleurs beaucoup.
Cependant, je souhaiterais m'orienter dans le domaine de l'IA et du Big Data, c'est pourquoi je choisirai la spécialité Valorisation des Connaissances pendant mon cursus.