\section*{Introduction} \markboth{Introduction}{}
\phantomsection
\addcontentsline{toc}{section}{Introduction}
\thispagestyle{noTitledHeader}
~\\
La Deutsche Bundesbank est la banque centrale de la République fédérale d'Allemagne. 
Son rôle principal est de maintenir la stabilité des prix et la valeur de l'euro. 
Elle est également chargée de superviser les banques et les institutions financières en Allemagne, 
d'assurer la sécurité et l'efficacité des paiements et des opérations financières, 
de gérer les réserves de change du pays, 
et de participer à la formulation et à la mise en œuvre de la politique monétaire européenne en tant que membre de l'Eurosystème.
\\

Elle est donc importante au niveau national, mais également international : 
\\
- Elle est membre de la Banque des règlements internationaux (BRI) et participe activement aux forums internationaux tels que le G20 et le Fonds monétaire international (FMI). 
\\
- Au niveau européen, elle est un membre important de la Banque centrale européenne (BCE) et joue un rôle clé dans la mise en œuvre de la politique monétaire de la zone euro. 
\\

La banque compte 30 filiales, comptant la centrale et les antennes régionales.
Ces dernières mettent en œuvre les missions de la banque centrale allemande au niveau régional et sont responsables de la coordination avec les acteurs économiques locaux et les autorités régionales pour favoriser le développement économique régional.

~Mon stage s'est déroulé à la Bundesbank Zentrale à Francfort-sur-le-Main, le siège principal de la banque. 
Il est le centre décisionnel et opérationnel de la banque et coordonne les activités des filiales régionales.
\\

~Je travaille dans le département DSZ (Datenservicezentrum) faisant partie du département Statistique, et dirigé par Stefan Bender. Son rôle est de fournir un accès aux données statistiques produites par la Bundesbank, 
ainsi qu'à d'autres sources de données pertinentes pour la politique monétaire et la supervision bancaire.
\\
Plus spécifiquement, la DSZ offre les services suivants :
\begin{itemize}
    \item Collecte et traitement des données provenant de différentes sources (banques centrales, agences gouvernementales et organisations internationales)
    \item Diffusion des données sous forme de tableaux, graphiques, rapports et bases de données interactives, accessibles aux décideurs politiques, chercheurs et analystes.
\end{itemize}
Il est composé de plusieurs sous-unités : 
\begin{itemize}
    \item DSZ 10 et 20 pour la recherche qui utilisent les données  collectées par les autres sous-unités
    \item DSZ 30 chargé des tâches de traitement de données avec Machine-Learning, Data Engineering et la mise en place de la plateforme de données interne
    \item DSZ 40 pour la collecte de données
    \item DSZ 1 (Sustainable Finance Data Hub) travaille en étroite collaboration avec la DSZ 40
\end{itemize}

Plus précisément dans la DSZ, je fais donc partie de l'équipe du Sustainable Finance Data Hub (SFDH), une unité spécialisée qui collecte et enrichit les données liées au climat pour soutenir la prise de décision optimale en matière de changement climatique. 
Il fournit un point d'accès central aux données, répond aux questions méthodologiques et prend part à des projets de génération de données innovants.


\bigbreak




% Groupe dans la situation :
% • internationale,
% • européenne,
% • française.
% Entreprise par rapport au 
% groupe.
% Service vous accueillant par 
% rapport à l'entreprise.