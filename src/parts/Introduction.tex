\section*{Introduction} \markboth{Introduction}{}
\phantomsection
\addcontentsline{toc}{section}{Introduction}
\thispagestyle{noTitledHeader}


Gaea21 est une association située à Genève à but non lucratif et un centre de recherche appliquée en développement durable, créé en 2005.
Sa raison d'être est la mise en oeuvre du développement durable et de l'agenda 21. Cela se traduit par plusieurs objectifs :
\begin{itemize}
    \item Sensibiliser et développer les compétences en Développement Durable, favoriser le changement
    de comportements 
    \item Stimuler la création d’emplois et d’entreprises vertes et durables
    \item Recenser, évaluer et soutenir les acteurs du Développement Durable
\end{itemize}

\bigbreak
Gaea21 crée donc des outils et applique des méthodes stimulant le changement de comportement et la transition
accélérée vers un modèle économique et social responsable.
Elle cible les ONG, individus/familles, entreprises et administrations/villes.
\bigbreak
La structure possède une antenne à Thonon-Les-Bains (Haute-savoie) depuis 2013, s'occupant principalement de la recherche et du développement en sciences physiques et naturelles.
Globalement, l'association a une portée plutôt régionale (Bassin genevois), mais les membres viennent du monde entier comme tout se fait en télétravail.
\bigbreak
Elle compte environ 200 membres et comprend divers départements :
\begin{itemize}
    \item Observatoire
    \item Administration \& Juridique 
    \item Formations
    \item Marketing
    \item Culture
\end{itemize} 

L'ensemble de ces départements travaille sur une quarantaine de projets.\\

Dans chacun d'eux, il y a 3 niveau de coordination : coordinateurs de Niveau 1 (gère le département ou sous-département), de niveau 2 (gère un projet composé de sous-projets), ou de niveau 3 (gère un sous-projet).

Le responsable et fondateur de l'association, Yvan Claude, est quant à lui coordinateur de niveau, et il suit tous les projets de tous les départements grâce à des réunions hebdomadaires.\\

Le drive est l'outil essentiel de l'assocation pour garder une trace de tous les documents.

Les réunions se font par Skype, ou Google Meet s'il y a peu de personnes.\\

Je fais partie du département Observatoire et du sous-département IT, qui s'occupe 
du design et de la réalisation des sites web ou applications mobiles pour les différents projets.

En IT, on compte une dizaine de membres, avec un coordinateur de Niveau 1 (mon tuteur), un coordinateur de Niveau 2, et des membres, travaillant sur un ou plusieurs sous-projets.
Si un sous-projet compte plusieurs membres, il y a un coordinateur de niveau 3.

Il dispose d'un dépôt Gitlab, dans lequel sont stockées tous les projets informatiques.



% Groupe dans la situation :
% • internationale,
% • européenne,
% • française.
% Entreprise par rapport au 
% groupe.
% Service vous accueillant par 
% rapport à l'entreprise.