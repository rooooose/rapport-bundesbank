\section{Déroulement du travail}

\subsection{Méthodologie et Organisation}

Durant toute la durée du stage, il y avait un suivi régulier, permettant une gestion de projet selon la méthode agile.

Chaque jour, une réunion d'alignement de 30 min s'appelant "Jour Fixe" avec mon équipe (dont mon tuteur), le SFDH, avait lieu.
Cela permettait à chaque membre de tenir les autres au courant de ce qu'il a fait la veille et de se faire aider par l'équipe en cas de questions.

Au début de chaque mois se déroulait additionnellement le Jour Fixe avec Stefan Bender, le responsable du département DSZ, afin d'aborder des points organisationnels et le tenir au courant du travail général de l'équipe du SFDH.

Un Jour Fixe DSZ se tenait tous les 3èmes mercredis du mois, avec l'ensemble du département, pour un alignement général des différents sous-départements. 

Au milieu du stage, un "Feedback Call" avec mon tuteur était l'occasion de partager mon ressenti sur le stage, ce que je voulais faire pendant les mois restants, si les projets précédents m'avaient plu, etc... 
Cela permettait aussi à mon tuteur de me donner son avis sur mon travail, et d'éventuellement m'indiquer les points à améliorer.
\\

Comme le travail avait lieu en hybride, toutes les réunions se faisaient en visio via Webex.
J'ai d'ailleurs bénéficié d'un ordinateur portable afin de pouvoir faire du télétravail plusieurs jours par semaine.
La communication avec les membres de la Bundesbank se faisait alors via la messagerie Jabber et par mail.
\\

Au niveau de l'organisation concernant purement mes tâches, le travail m'était toujours donné tâche par tâche, ce qui me permettait de ne pas être débordée et d'être aussi plus efficace.
De plus, cela encourage des réunions fréquentes et ainsi une gestion de projet agile.

Dès qu'une tâche était terminée, j'organisais une réunion d'avancement avec les personnes concernées pour avoir leur feedback ainsi que de nouvelles tâches.

\paragraph{Sécurité informatique}

S'agissant d'une grande banque centrale, la sécurité est bien sûr prioritaire. 
Les ordinateurs ne peuvent être connectés à internet seulement lorsqu'ils sont connectés au réseau de la Banque (directement ou par VPN), et seuls certains sites sont autorisés.

La Banque possède un serveur interne dans lequel tous les dossiers et fichiers sont partagés. 
Chaque dossier est protégé et n'est accessible qu'à certains groupes.

Si un nouveau logiciel a besoin d'être installé, il doit être commandé via la plateforme Servity, puis l'installation doit être déclenchée par le service IT.
De même pour certains accès, notamment pour Gitlab ou Python, qui sont à demander sur la plateforme BIAM.
\\

La mise en place de Python comprend notamment la création d'environnements virtuels pour chaque projet, afin d'éviter des problèmes aux changements de version de Python.

Heureusement, tout cela est très bien guidé sur Confluence, une plateforme de documentation partagée, où sont notamment disponibles des guides pour la mise en place de Python sur les machines de la banque.
Elle est également utilisée pour documenter certains projets avec l'état des lieux actuel, les recherches déjà effectuées, les étapes suivantes, etc...

% \pagebreak
\subsection{Application de la méthode et Résultats}

\subsubsection{Projet Gaia}


\subsubsection{Projet ESCB Exchange}

\subsubsection{Projet CSDB}

\subsubsection{Projet NFIG}

\subsubsection{Projet de recherche NLP}

\subsection{Planning général suivi}