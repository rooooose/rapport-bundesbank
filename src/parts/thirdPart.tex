\section{Déroulement du travail}

\subsection{La méthodologie}
~\\

Durant toute la durée du stage, il y avait un suivi régulier, permettant une gestion de projet selon la méthode agile.\\

Ainsi, cela fonctionnait par cycles répétés :

Tous les jeudis, une réunion avec l'ensemble du département IT et le client (Yvan, le responsable de l'association) avait lieu afin de recueillir ses besoins pour le site.\\
Si nécessaire, un poker planning était réalisé pour répartir les tâches efficacement selon les difficultés et le temps prévu.\\

Tous les jours, une réunion de suivi avec le tuteur et l'équipe permettait de partager notre avancée et obtenir de l'aide sur des points de blocage.\\

Tous les mardis, une réunion avec le département Design permettait de mettre au clair certains points ou de faire des commandes, car le développement d'un site est étroitement lié au design.\\

Une fois toutes les 2 semaines, une réunion transversale pour le projet Répertoire Vert avait lieu.

Pour finir, tous les lundi et jeudi, une mise en ligne du site était nécessaire pour garder la version en ligne toujours à jour et présentable au client.

D'un point de vue technique, il fallait mettre à jour le site sur un serveur Linux distant à l'aide de Git, un logiciel de gestion de versions que j'ai particulièrement utilisé durant ce stage. \\

Justement, il était important de faire des \textbf{commits et push} au moins 1 fois par jour, pour garder un contrôle des versions du site.
Chaque membre de l'équipe possédait une branche à son nom. Plusieurs fois par semaine, un merge des toutes les branches sur la branche de développement était réalisé, avant de faire un merge sur master de la branche de Développement.
En effet, la branche master est celle utilisée pour la mise en production, elle doit donc être propre, fonctionnelle et régulièrement mise à jour. \\

D'autre part, il ne fallait pas négliger le remplissage les outils de suivi, à savoir : \\
- \textbf{le journal de bord} pour les heures et les tâches réalisées par jour/semaine/mois\\ 
- \textbf{le tableau des tâches} pour organiser les tâches à faire/en cours/faites, et répertorier le temps mis pour chacune d'entre elles.\\
(voir Annexes)\\

Des points début, mi-stage et fin de stage étaient réalisés avec le tuteur et la coach RH pour vérifier le remplissage des outils et voir si tout allait bien.

\pagebreak
\subsection{Application de la méthode et Résultats}

\subsubsection{Projet Gaia}


\subsubsection{Projet ESCB Exchange}

\subsubsection{Projet CSDB}

\subsubsection{Projet NFIG}

\subsubsection{Projet de recherche NLP}

\subsection{Planning général suivi}