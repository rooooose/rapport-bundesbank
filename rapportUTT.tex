% !TeX root = ./rapportUTT.tex
% !BIB TS-program = biber

\documentclass{rUTT}
% Pour retirer le thème couleur UTT,
%   Commenter la ligne précédente
%   Décommenter la ligne dessous
% \documentclass[noUTTcolors]{rUTT}


%%%%%%%%%%%%%%%%%%%%%%%%%%%%%%%%%%%%%%%%%%%%%%%%%%%%%%%%%%%%
% Quelques trucs à savoir pour modifier en paix :
% TOC = Table of Contents
% LOF / LOT = List Of Figures / List Of Tables
%%%%%%%%%%%%%%%%%%%%%%%%%%%%%%%%%%%%%%%%%%%%%%%%%%%%%%%%%%%%
% Si on est on mode rapport de stage :
\RPeda{{\sc NIGRO} Jean-Marc} % Responsable pédagogique

\Entreprise{Gaea21}
\Lieu{Avenue des Morgines 9, 1213 Petit-Lancy, Suisse}
\REntre{Yvan CLAUDE}

% Mots clés du Thésaurus
\Kone{Mise en place, mise en oeuvre} % Nature de l'activité
\Ktwo{Associations} % Branche d'activité économique
\Kthree{Informatique} % Domaines technologiques
\Kfourth{Gestion - Logiciels} % Application Physique directe
%%%%%%%%%%%%%%%%%%%%%%%%%%%%%%%%%%%%%%%%%%%%%%%%%%%%%%%%%%%%

\Semestre{Automne 2021}
%\UE{LT01} %Nom de l'UE OU nom complet de la branche si en mode rapport de stage !
\UE{Informatique et Systèmes d'Information}

% Le titre de votre rapport OU le résumé de votre stage si en mode stage
%\title{Un rapport en \LaTeX \\ écrit avec amour}

\title{Mon stage s'est déroulé au sein de l'association Gaea21 en tant que bénévole, dans le département IT. \\
Ma mission a consisté au développement Full-Stack du site web du projet "Répertoire Vert", à l'aide des frameworks Symfony et ReactJS. En tant que chef de projet, je devais également co-gérer l'équipe projet.
\\ La gestion du projet se faisait selon la méthode Agile, avec différentes phases répétées cycliquement : \\
Recueil des besoins du responsable de l'association, Mise en place d'un planning et répartition des tâches dans l'équipe, Mise en oeuvre, Réunion hebdomadaire avec le responsable et l'équipe et Déploiement de la version terminée du site.
\\ Le but est de livrer un site web fonctionnel permettant le référencement et l’évaluation de tous les
produits et/ou services professionnels verts proposés
dans une région définie, destiné aussi bien aux entreprises qu'aux particuliers.}

%% En mode Année - Mois - Jour
\date{\today} % Pour la date de compilation
% \date{2038-01-19}

\author{{\sc FRANK} Shir
% \and
% {\sc Nom} Prénom
% \break
% {\sc Nom} Prénom
% \and
% {\sc Nom} Prénom
% \break
% {\sc Nom} Prénom
}

% Texte affiché sur le carré bleu
%\newcommand{\titletext}{Vous êtes probablement assez bon pour travailler dans cette entreprise pour laquelle vous pensez ne pas être assez bon.}

\newcommand{\titletext}{Développeuse Web FullStack \\ Symfony / ReactJS}

% n'oubliez pas de changer le language principal dans rUTT.cls
% en options du package babel !
\selectlanguage{french}

\begin{document}
    %%%% - Choix de la page de garde
    %\frontpagereports % Pour le modèle rapports de TDs / TPs / Projets
    \frontpageSTB % Pour le modèle rapports de Stages
    %\frontpageSTC % Pour le modèle rapports de Stages des TC

    \myemptypage % page blanche après page de garde pour impression recto verso

    % Ici on organise nos parties
    \justify % on justifie notre texte via ragged2e


    \pagenumbering{gobble} % on n'affiche pas les numéros de page
    \import{src/parts/}{Remerciements.tex} % Toujours avant le sommaire !

    \clearpage

    % Le Sommaire
    \shorttoc{Sommaire}{2}

    \clearpage

    \pagenumbering{arabic}

    \import{src/parts/}{Introduction.tex}

    \clearpage

    \import{src/parts/}{firstPart.tex}

    \clearpage

    \import{src/parts/}{secondPart.tex}

    % \clearpage

    % \import{src/parts/}{thirdPart.tex}

    % \clearpage

    % \import{src/parts/}{scienticPart.tex}

    \clearpage

    %\tripleS

    \import{src/parts/}{Conclusion.tex}

    \clearpage

    % Bibliographie !

    {
    \pagenumbering{gobble} % On n'affiche pas les numéros de page
    \phantomsection % hyperlinks will target the correct page
    \markboth{Bibliographie}{}
    \raggedright % pour éviter certaines erreurs rares d'affichage
    \sloppy
    \nocite{*} % pour faire apparaître tout du fichier bib
    \printbibliography[title={Bibliographie},heading=bibintoc]

    \clearpage
    \listoffigures
    \addcontentsline{toc}{section}{\listfigurename}
    \listoftables
    \addcontentsline{toc}{section}{\listtablename}
    }
    \clearpage

    \pagenumbering{Roman} % On numérote en romain pour les annexes

    % Annexes !
    \import{src/parts/Annexes/}{Annexes.tex}

    \clearpage

    
    % Toujours avoir la table des matières en dernier !
    % Ici figure tout, même les annexes
    % Commenter/supprimer pour enlever la table des matières
    % \thispagestyle{empty}
    % \setcounter{tocdepth}{10} % Profondeur de la table des matières
    % \tableofcontents

    % On laisse une page blanche à la fin pour l'impression, c'est plus joli
    \clearpage
    \myemptypage

\end{document}
