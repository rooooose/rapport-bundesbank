%!TeX root = ./rapportUTT.tex
%!BIB TS-program = biber

\documentclass{rUTT}
% Pour retirer le thème couleur UTT,
%   Commenter la ligne précédente
%   Décommenter la ligne dessous
% \documentclass[noUTTcolors]{rUTT}


%%%%%%%%%%%%%%%%%%%%%%%%%%%%%%%%%%%%%%%%%%%%%%%%%%%%%%%%%%%%
% Quelques trucs à savoir pour modifier en paix :
% TOC = Table of Contents
% LOF / LOT = List Of Figures / List Of Tables
%%%%%%%%%%%%%%%%%%%%%%%%%%%%%%%%%%%%%%%%%%%%%%%%%%%%%%%%%%%%
% Si on est on mode rapport de stage :
\RPeda{{\sc MATTA} Nada} % Responsable pédagogique

\Entreprise{Deutsche Bundesbank}
\Lieu{Mainzer Landstraße 16, 60325 Francfort-sur-le-Main, Allemagne}
\REntre{Stefan BENDER}

% Mots clés du Thésaurus
\Kone{Mise en place, mise en oeuvre} % Nature de l'activité
\Ktwo{Services des organismes financiers} % Branche d'activité économique
\Kthree{Informatique} % Domaines technologiques
\Kfourth{Analyse des données - Logiciels} % Application Physique directe
%%%%%%%%%%%%%%%%%%%%%%%%%%%%%%%%%%%%%%%%%%%%%%%%%%%%%%%%%%%%

\Semestre{Printemps 2023}
%\UE{LT01} %Nom de l'UE OU nom complet de la branche si en mode rapport de stage !
\UE{ISI - VDC}

% Le titre de votre rapport OU le résumé de votre stage si en mode stage
%\title{Un rapport en \LaTeX \\ écrit avec amour}

\title{Mon stage s'est déroulé au sein de la Deutsche Bundesbank en tant que Data Scientist dans le centre de service des données. \\
Mes missions ont consisté à analyser et exploiter de gros volumes de données avec Python et R afin d'en ressortir une valeur spécifique, qu'elle concerne l'optimisation du workflow ou l'augmentation de la qualité des données. La collecte de données faisait aussi partie intégrante du stage.
Les différents projets concernaient les tâches suivantes : \\
- Crawling des résultats de recherche Google\\
- Comparaison de gros jeux de données et optimisation\\
- Analyses de gros jeux de données\\
- Génération automatisée d'un rapport de données\\
- Travail avec ChatGPT et optimisation de l'input\\
L'enjeu est de valoriser au maximum les données du service en en générant de nouvelles qui soient compréhensibles et utilisables par les différents corps de métier présents à la Bundesbank.}

%% En mode Année - Mois - Jour
\date{\today} % Pour la date de compilation
% \date{2038-01-19}

\author{{\sc FRANK} Shir
% \and
% {\sc Nom} Prénom
% \break
% {\sc Nom} Prénom
% \and
% {\sc Nom} Prénom
% \break
% {\sc Nom} Prénom
}

% Texte affiché sur le carré bleu
%\newcommand{\titletext}{Vous êtes probablement assez bon pour travailler dans cette entreprise pour laquelle vous pensez ne pas être assez bon.}

\newcommand{\titletext}{Data Science en R/Python au Centre de Services des Données}

% n'oubliez pas de changer le language principal dans rUTT.cls
% en options du package babel !
\selectlanguage{french}

\begin{document}
    %%%% - Choix de la page de garde
    %\frontpagereports % Pour le modèle rapports de TDs / TPs / Projets
    \frontpageSTB % Pour le modèle rapports de Stages
    %\frontpageSTC % Pour le modèle rapports de Stages des TC

    \myemptypage % page blanche après page de garde pour impression recto verso

    % Ici on organise nos parties
    \justify % on justifie notre texte via ragged2e


    \pagenumbering{gobble} % on n'affiche pas les numéros de page
    \import{src/parts/}{Remerciements.tex} % Toujours avant le sommaire !

    \clearpage

    % Le Sommaire
    % \shorttoc{Sommaire}{2}
    \tableofcontents

    \clearpage

    \pagenumbering{arabic}

    \import{src/parts/}{Introduction.tex}

    % \clearpage

    \import{src/parts/}{firstPart.tex}

    \clearpage

    \import{src/parts/}{secondPart.tex}

    \clearpage

    \import{src/parts/}{thirdPart.tex}

    % \clearpage

    % \import{src/parts/}{scienticPart.tex}

    \clearpage

    %\tripleS

    \import{src/parts/}{Conclusion.tex}

    \clearpage

    % Bibliographie !

    {
    \pagenumbering{gobble} % On n'affiche pas les numéros de page
    \phantomsection % hyperlinks will target the correct page
    \markboth{Bibliographie}{}
    \raggedright % pour éviter certaines erreurs rares d'affichage
    \sloppy
    \nocite{*} % pour faire apparaître tout du fichier bib
    \printbibliography[title={Bibliographie},heading=bibintoc]

    \clearpage
    \listoffigures
    \addcontentsline{toc}{section}{\listfigurename}
    % \listoftables
    % \addcontentsline{toc}{section}{\listtablename}
    }
    \clearpage

    \pagenumbering{Roman} % On numérote en romain pour les annexes

    % Annexes !
    % \import{src/parts/Annexes/}{Annexes.tex}

    \clearpage

    
    % Toujours avoir la table des matières en dernier !
    % Ici figure tout, même les annexes
    % Commenter/supprimer pour enlever la table des matières
    % \thispagestyle{empty}
    % \setcounter{tocdepth}{10} % Profondeur de la table des matières
    

    % On laisse une page blanche à la fin pour l'impression, c'est plus joli
    \clearpage
    \myemptypage

\end{document}
